\section{Теорема Слупецкого. Замечание Яблонского о возможности сужения множества одноместных функций. Теорема Мартина}

\begin{theorem}[Слупецкий]
    Пусть система $F$ функций $k$-значной логики, где $k \geqslant 3$, содержит все функции одной переменной. Тогда для полноты $F$ необходимо и достаточно, чтобы она содержала существенную функцию, принимающую все $k$ значений.
\end{theorem}

В доказательстве под <<леммой $j$>> подразумевается лемма из вопроса номер $24 + j$ ($j = 1, 2, 3$).

\begin{proof}
    $\Rightarrow$. Пусть $F$ не содержит существенной функции, принимающей все $k$ значений. Покажем, что операции суперпозиции не позволяют получить такую функцию из функций, которые либо не принимают все $k$ значений, либо не являются существенными:
    \begin{enumerate}
        \item \textbf{Операция подстановки переменных}. Пусть $g(x_1, \ldots, x_n) \vcentcolon = f(x_{i_1}, \ldots, x_{i_n})$. Если функция $f$ не принимает все $k$ значений, то $g$ тоже не будет их принимать. Если $f$ имела единственную существенную переменную $x_m$, то $g$ будет иметь единственную существенную переменную $x_{i_m}$.
        \item \textbf{Операция подстановки одной функции в другую}. Пусть функция $h(x_1, \ldots, x_{n + m - 1})$ определена как $f(x_1, \ldots, x_{n - 1}, g(x_n, \ldots, x_{n + m - 1}))$. Если $f$ не принимает все $k$ значений, то и $h$ их не принимает. Если $f$ принимает все $k$ значений, то она должна иметь единственную существенную переменную. Если её существенная переменная --- не $x_n$, то она же будет единственной существенной переменной у $h$. Если единственной существенной переменной функции $f$ служит $x_n$, то рассматриваем функцию $g$. Если она принимает не все $k$ значений, то и $h$ не будет принимать все $k$ значений. Если функция $g$ принимает все $k$ значений, то она имеет единственную существенную переменную. Но тогда она окажется единственной существенной переменной и для $h$.
        \item \textbf{Операция добавления либо удаления фиктивной переменной}. Очевидно.
    \end{enumerate}

    Т.\,к. суперпозициями из $F$ нельзя получить существенной функции, принимающей все $k$ значений, то она неполна. Ведь такие функции существуют, например, $\max(x_1, x_2)$.

    $\Leftarrow$. Пусть $F$ имеет существенную функцию $f(x_1, \ldots, x_n)$, принимающую все $k$ значений. По лемме 3 существует квадрат
    \[
        \left\{(\alpha_1, \ldots, \alpha_{i - 1}, x, \alpha_{i + 1}, \ldots, \alpha_{j - 1}, y, \alpha_{j + 1}, \ldots, \alpha_n) : x \in \{p_1, p_2\}, y \in \{q_1, q_2\}\right\},
    \]
    на котором функция $f$ принимает некоторое своё значение ровно в одной точке. Обозначим это значение $a$. Рассмотрим вспомогательную функцию $\varphi_0(x)$, равную нулю при $x = a$ и равную $1$ в остальных случаях. Т.\,к. она зависит от одной переменной, она принадлежит множеству $F$. Поэтому следующая функция
    \[
        g(x_1, x_2) \vcentcolon = \varphi_0(f(\alpha_1, \ldots, \alpha_{i - 1}, x_1, \alpha_{i + 1}, \ldots, \alpha_{j - 1}, x_2, \alpha_{j + 1}, \ldots, \alpha_n))
    \]
    получена суперпозициями над $F$.

    Функция $g$ на квадрате $\{(p_1, q_1), (p_1, q_2), (p_2, q_1), (p_2, q_2)\}$ принимает значения $0$ и $1$, причём значение $0$ она принимает в единственной точке. Без ограничения общности можно считать, что $g(p_1, q_1) = 0$. В остальных точках квадрата функция $g$ равна $1$.

    Введём вспомогательную функцию $\varphi_1(x)$, равную $p_1$ при $x = 0$ и равную $p_2$ в остальных случаях. Введём также функцию $\varphi_2(x)$, равную $q_1$ при $x = 0$ и равную $q_2$ в остальных случаях. Обе эти функции зависят от одной переменной и принадлежат $F$.

    Поэтому функция $g^\prime(x_1, x_2) \vcentcolon g(\varphi_1(x_1), \varphi_2(x_2))$ получается суперпозициями над $F$. Она обращается в $0$ при $x_1 = x_2 = 0$, а в остальных случаях она равна $1$. Таким образом, данная функция совпадает с дизъюнкцией, если значения её аргументов ограничить множеством $\{0, 1\}$. Будем обозначать её $x_1 \vee_{01} x_2$.

    Функция $j_i(x)$, равная $1$ при $x = i$ и равна $0$ в остальных случаях, зависит от одной переменной, а потому входит в $F$. В частности, функция $j_0(x)$, совпадающая с отрицанием на множестве $\{0, 1\}$, входит в $F$.

    Используя законы де Моргана, нетрудно теперь получить функцию в $P_k$, ограничение которой на множество $\{0, 1\} \times \{0, 1\}$ совпадает с конъюнкцией: $g^{\prime\prime}(x_1, x_2) = j_0(j_0(x_1) \vee_{01} j_0(x_2))$. Будем обозначать эту функцию через $x_1 \&_{01} x_2$.

    Пусть теперь $h(x_1, \ldots, x_m)$ --- произвольная функция в $P_k$, принимающая только значения $0$ и $1$. Покажем, что её можно получить суперпозициями над $F$. Используем обобщение совершенной дизъюнктивной нормальной формы для $P_k$:
    \[
        h(x_1, \ldots, x_m) = \bigvee_{(\sigma_1, \ldots, \sigma_m)}j_{\sigma_1}(x_1) \& \ldots \& j_{\sigma_m}(x_m) \& h(\sigma_1, \ldots, \sigma_m).
    \]

    Здесь символом $\bigvee$ обозначен максимум, символом $\&$ --- минимум. Максимум берётся по всем наборам $(\sigma_1, \ldots, \sigma_m)$ элементов $E_k$. Заметим, что все значения $j_{\sigma_1}(x_i)$ и $h(\sigma_1, \ldots, \sigma_m)$ равны $0$ либо $1$. Поэтому в выражении для $h$ можно вместо максимума и минимума использовать $\vee_{01}$ и $\&_{01}$:
    \[
        h(x_1, \ldots, x_m) = \underset{{(\sigma_1, \ldots, \sigma_m)}}{\bigvee{_{01}}}j_{\sigma_1}(x_1) \&_{01} \ldots \&_{01} j_{\sigma_m}(x_m) \&_{01} h(\sigma_1, \ldots, \sigma_m).
    \]
    
    С учётом того, что константы $h(\sigma_1, \ldots, \sigma_m)$ принадлежат $F$, приведённая формула выражает $h$ через функции, которые были получены суперпозициями над $F$. Таким образом, любая функция из $P_k$, принимающая только значения $0$ и $1$, получается суперпозициями над $F$. Если теперь $h(x_1, \ldots, x_m)$ --- функция из $P_k$, принимающая только значения $a$ и $b$, то можно рассмотреть функцию $h^\prime(x_1, \ldots, x_m)$, принимающую значение $0$, если $h(x_1, \ldots, x_m) = a$, и значение $1$ в противном случае. Рассмотрим также принадлежащую $F$ функцию $\psi(x)$, равную $a$, если $x = 0$, и равную $b$ в противном случае. Очевидно, $h(x_1, \ldots, x_m) = \psi(h^\prime(x_1, \ldots, x_m))$, причём как $h^\prime$, так и $\psi$ получены суперпозициями над $F$. Это означает, что любая функция в $P_k$, принимающая не более двух значений, получается суперпозициями над $F$.

    Доказанное утверждение можно рассматривать как базу индукции. Предположим теперь, что для некоторого $\ell$, $3 \leqslant \ell \leqslant k$, установлено, что все функции из $P_k$, принимающие не более, чем $\ell - 1$ значение, получаются суперпозициями над $F$. Покажем, что это также верно и для всех функций, принимающих не более чем $\ell$ значений.

    Снова рассмотрим в $F$ существенную функцию $f(x_1, \ldots, x_n)$, принимающую $k$ значений. По лемме 2, существуют подмножества $G_1, \ldots, G_n$ множества $E_k$ такие, что $1 \leqslant \abs{G_i} \leqslant \ell - 1$ $\forall i = 1, \ldots, n$, причём на множестве $G_1 \times \ldots \times G_n$ функция $f$ принимает хотя бы $\ell$ значений. Обозначим эти $\ell$ значений через $a_1, \ldots, a_\ell$. Рассмотрим наборы из $G_1 \times \ldots \times G_n$, на которых функция $f$ принимает данные значения:
    \begin{align*}
        a_1 &= f(a_{11}, \ldots, a_{1n}),\\
        \cdots\\
        a_\ell &= f(a_{\ell 1}, \ldots, a_{\ell n}).
    \end{align*}

    Заметим, что $\{a_{11}, \ldots, a_{\ell 1}\} \subseteq G_1, \ldots, \{a_{1n}, \ldots, a_{\ell n}\} \subseteq G_n$.

    Возьмём теперь произвольную функцию $h(x_1, \ldots, x_m)$ из $P_k$, принимающую только значения $a_1, \ldots, a_\ell$, и попробуем подобрать вспомогательные функции $\psi_1, \ldots, \psi_n$ так, чтобы имело место тождество $h(x_1, \ldots, x_m) = f(\psi_1(x_1, \ldots, x_m), \ldots, \psi_n(x_1, \ldots, x_m))$. Будем делать это, используя указанные выше $\ell$ равенств.

    Рассмотрим произвольный набор $(\alpha_1, \ldots, \alpha_m)$ значений переменных $x_1, \ldots, x_m$. На этом наборе функция $h$ принимает какое-то значение $a_i$. Определим значения $\psi_1(\alpha_1, \ldots, \alpha_m), \ldots, \psi_n(\alpha_1, \ldots, \alpha_m)$ равными, соответственно, $a_{i1}, \ldots, a_{in}$. Тогда получим $f(\psi_1(\alpha_1, \ldots, \alpha_m), \ldots, \psi_n(\alpha_1, \ldots, \alpha_m)) = a_i$.

    Таким образом, значения функций $\psi_1, \ldots, \psi_n$ определены для всех наборов значений их аргументов. При этом желаемое тождество выполнено по построению.

    Заметим, что функция $\psi_i$ принимает в качестве своих значений только элементы $\{a_{1i}, \ldots, a_{\ell i}\}$, которые принадлежат множеству $G_i$. Поэтому данная функция принимает не более чем $\ell - 1$ значение, и по предположению индукции получается суперпозициями над $F$. Ввиду указанного тождества, функция $h$ тоже получается суперпозициями над $F$.

    Итак, любая функция из $P_k$, принимающая только значения $a_1, \ldots, a_\ell$, получается суперпозициями над $F$. Если $\ell = k$, то это уже означает, что любая функция из $P_k$ получается суперпозициями над $F$.

    Пусть $\ell < k$. Рассмотрим произвольную функцию $h$ из $P_k$, принимающую не более чем $\ell$ значений. Пусть её значения принадлежат списку $b_1, \ldots, b_\ell$. Рассмотрим функцию $h^\prime$, значение которой на произвольном наборе получается из значения функции $h$ на том же наборе заменой $b_i$ на $a_i$. Функция $h^\prime$ получается суперпозициями над $F$. Определим функцию $\varphi(x)$, принмающую в точке $a_i$ значение $b_i$, а в остальных точках равную $0$. Легко видеть, что $h(x_1, \ldots, x_m) = \psi(h^\prime(x_1, \ldots, x_m))$, т.\,е. $h$ получается суперпозициями над $F$.

    Это завершает доказательство шага индукции. При $\ell = k$ получаем, что система $F$ полна.
\end{proof}

Заметим, что все одноместные функции, которые мы использовали, принимали не более $k - 1$ значений. Заметим, что <<переходники>> $\psi$, переводящие $a_i$ в $b_i$, нужны были лишь при $\ell < k$. Отсюда получается следующий результат:

\begin{theorem}[Яблонский]
    Пусть система $F$ функций $k$-значной логики, где $k \geqslant 3$, содержит все функции одной переменной, принимающие не более $k - 1$ значения. Тогда для полноты $F$ необходимо и достаточно, чтобы она содержала существенную функцию, принимающую все $k$ значений.
\end{theorem}

\begin{theorem}[Мартин]
    Функция $f(x_1, \ldots, x_n)$ из $P_k$, образует полную систему тогда и только тогда, когда она порождает все функции одной переменой, принимающие не более чем $k - 1$ значение.
\end{theorem}

\begin{proof}
    $\Rightarrow$. Очевидно.
    
    $\Leftarrow$. Пусть $f$ порождает все функции одной переменной, принимающие не более чем $k - 1$ значение. В частности, она порождает все константы. Поэтому $f$ должна принимать все $k$ значений. Предположим, что она не является существенной. Тогда она должна иметь ровно одну существенную переменную (отсутствие существенные переменных делало бы функцию константой, и она не могла бы принимать $k$ значений). Обозначим через $M(k)$ класс всех функций в $P_k$, принимающих $k$ значений и имеющих одну существенную переменную. Фактически, это перестановки с добавлением несущественных переменных. Заметим, что ничего кроме таких же перестановок суперпозициями из $f$ получить нельзя. Действительно, проверим для каждой операции суперпозиции:

    \begin{enumerate}
        \item \textbf{Операция подстановки переменных}. Пусть $g(x_1, \ldots, x_n) \vcentcolon = f(x_{i_1}, \ldots, x_{i_n})$, $f \in M(k)$. Если $f$ имела существенную переменную $x_j$, то $g$ будет иметь единственную существенную переменную $x_{i_j}$. Варьируя значение этой переменной в правой части равенства, будем получать все $k$ значений функции $f$, но тогда и $g$ принимает все $k$ значений. Селдвоательно, $g \in M(k)$.
        \item \textbf{Операция подстановки одной функции в другую}. Пусть функция $h(x_1, \ldots, x_{n + m - 1})$ определена как $f(x_1, \ldots, x_{n - 1}, g(x_n, \ldots, x_{n + m - 1}))$, причём $f, g \in M(k)$. Если $f$ имеет существенную переменную $x_i$, где $i < n$, то и $h$ будет иметь единственную существенную переменную $x_i$. Варьируя значение этой переменной, будем получать в правой части равенства все $k$ значений. Следовательно, $h \in M(k)$.

            Если $f$ имеет существенную переменную $x_n$, то рассматриваем единственную существенную переменную функции $g$. Тогда задание её значения однозначно определяет значение $g(x_n, \ldots, x_{n + m - 1})$, а вслед за этим и значение $f(x_1, \ldots, x_{n - 1}, g(x_n, \ldots, x_{n + m - 1}))$. Таким образом, она является единственной существенной переменной функции $h$. Варьируя значение данной переменной, можно получить любые $k$ значений функции $g$, а значит, любые $k$ значений функции $f$, т.\,е. любые $k$ значений функции $h$. Следовательно, $h \in M(k)$.
        \item \textbf{Операция добавления либо удаления фиктивной переменной}. Очевидно.
    \end{enumerate}

    Таким образом, из функции $f$ можно получить суперпозициями только функции с единственной существенной переменной, принимающие все $k$ значений. Но тогда нельзя получить ни одной константы. Полученное противоречие означает, что функция $f$ должна быть существенной. И тогда, по теореме Яблонского, она образует полную систему.
\end{proof}

