\section{Существование для неполной системы $F$ такого множества $K$, что $V_k \notin K$ и $F \subseteq U(K)$}

\begin{lemma}
    Если система $F$ функций $k$-значной логики неполна, то в $P_k$ существует множество $K$ функций от двух переменных, содержащее обе селекторные функции и не содержащее функции Вебба такое, что $F \subseteq U(K)$.
\end{lemma}

\begin{proof}
    Введём для $F$ какую-либо сигнатуру $\Sigma$ и рассмотрим последовательность Кузнецова $G_1, G_2, \ldots$ Пусть $G_m = G_{m + 1} = \ldots$ Т.\,к. $F$ неполна, то $V_k \notin G_m$. Положим $K \vcentcolon = G_m \cup \{g_1^2, g_2^2\}$. Очевидно, $K$ всё ещё не содержит функции Вебба $V_k$.

    Покажем, что $F$ сохраняет $K$. Пусть $F(\widetilde{x}^n) \in F$; $h_1, \ldots, h_n \in K$. Рассмотрим функцию
    \[
        h(x_1, x_2) \vcentcolon = f(h_1(x_1, x_2), \ldots, h_n(x_1, x_2)).
    \]

    Пусть $s$ --- символ, обозначающий функцию $f$ в сигнатуре $\Sigma$. Если $h_j(x_1, x_2) \in G_m$, то она определяется некоторой формулой $A_j$ в сигнатуре $\Sigma$, глубина которой меньше $m$. Если $h_j(x_1, x_2)$ --- селекторная функция $g^2_1(x_1, x_2)$, то возьмём в качестве $A_j$ переменную $x_1$; если $h_j(x_1, x_2)$ --- селекторная функция $g^2_2(x_1, x_2)$, то возьмём в качестве $A_j$ переменную $x_2$. Тогда функция $h$ будет определяться формулой $s(A_1, \ldots, A_n)$, глубина которой меньше $m + 1$, так что реализуемая ею функция $h(x_1, x_2)$ принадлежит $G_{m + 1}$. Однако, $G_{m + 1} = G_m \subseteq K$, и $h \in K$. Таким образом, $F$ сохраняет $K$, и $F \subseteq U(K)$.
\end{proof}

