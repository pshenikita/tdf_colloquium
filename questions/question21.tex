\section{Селекторные функции. Сохранение множества $K$, включающего селекторные функции. Описание класса $S$ как класса сохранения некоторого множества $K$. Замкнутость класса $U(K)$ всех функций, сохраняющих $K$}

\begin{definition}
    Функции $g_i^p(x_1, \ldots, x_p) = x_i$, $i = 1, \ldots, p$, называются \textit{селекторными функциями}. Они выбирают заданный разряд из набора значений аргументов.
\end{definition}

\begin{definition}
    Пусть $K$ --- некоторое множество функций $h(x_1, \ldots, x_p)$ из $P_k$, зависящих от заданного числа переменных $p$ и содержащее все селекторные функции от $p$ переменных. Скажем, что функция $f(x_1, \ldots, x_n)$ \textit{сохраняет множество $K$}, если для любых функций $h_1, \ldots, h_n$ из $K$ выполнено
    \[
        f(h_1(x_1, \ldots, x_p), \ldots, h_n(x_1, \ldots, x_p)) \in K.
    \]
\end{definition}

\begin{example}
    Пусть $k = 2$, $p = 1$, $K = \{x, \overline{x}\}$. Иными словами, в $K$ входят функци $x^\sigma$, $\sigma \in \{0, 1\}$. Свойство сохранения множества $K$ функцией $f$ означает, что для любых $\sigma_1, \ldots, \sigma_n$ из $E_2$ имеет место $f(x^{\sigma_1}, \ldots, x^{\sigma_n}) = x^\sigma$ при некотором $\sigma$. Подставим в это тождество поочерёдно значения $1$:
    \[
        f(1^{\sigma_1}, \ldots, 1^{\sigma_n}) = 1^\sigma \Leftrightarrow f(\sigma_1, \ldots, \sigma_n) = \sigma
    \]
    и $0$:
    \[
        f(0^{\sigma_1}, \ldots, 0^{\sigma_n}) = 0^\sigma \Leftrightarrow f(\overline{\sigma_1}, \ldots, \overline{\sigma_n}) = \overline{\sigma}.
    \]

    Объединяя полученные равенства, имеем $f(\overline{\sigma_1}, \ldots, \overline{\sigma_n}) = \overline{f}(\sigma_1, \ldots, \sigma_n)$. Таким образом, свойство сохранения класса $K$ равносильно свойству самодвойственности функции.
\end{example}

Пусть $K$ --- некоторое множество функций $h(x_1, \ldots, x_p)$ из $P_k$, зависящих от заданного числа переменных $p$ и содержащее все селекторные функции от $p$ переменных. Через $U(K)$ обозначим множество всех функций в $P_k$, сохраняющих $K$.

\begin{lemma}
    Класс $U(K)$ замкнут.
\end{lemma}

\begin{proof}
    Лемма доказывается точно так же, как доказывалось утверждение о замкнутости класса $T_0$. Рассматриваются операции суперпозиции: операция подстановки переменных, операция подстановки одной функции в другую, а также операция добавления и удаления фиктивных переменных. В каждом случае свойство новой функции сохранять $P_k$ очевидным образом вытекает из этого свойства для исходных функций.
\end{proof}

