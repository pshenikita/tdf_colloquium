\section{Полные системы в $P_k$. Примеры. Система $\{\max(x_1, x_2), \overline{x}\}$. Функция Вебба}

\begin{definition}
    Система $F \subseteq P_k$ назовём \textit{полной}, если каждая функция из $P_k$ получается из $F$ суперпозициями.
\end{definition}

\begin{example}
    Полными в $P_k$ являются следующие системы:
    \begin{enumerate}
        \item $P_k$ --- очевидно;
        \item $\{0, 1, \ldots, k - 1, J_0(x), \ldots, J_{k - 1}(x), \min(x_1, x_2), \max(x_1, x_2)\}$ --- эта система образована функциями, использованными в приведённом выше аналоге СДНФ.
    \end{enumerate}
\end{example}

\begin{theorem}
    Система $\{\overline{x}, \max(x_1, x_2)\}$ полна в $P_k$.
\end{theorem}

\begin{proof}
    Прежде всего покажем, что через функции данной системы можно выразить все константы. Из $\overline{x} = x + 1$ получаем $x + 2, x + 3, \ldots, x + k - 1$, навешивая каждый раз нужное количество отрицаний Поста. Т.\,к. значения $x, x + 1, \ldots, x + k - 1$ различны по модулю $k$, а их число равно $k$, то они покрывают все значения $0, 1, \ldots, k - 1$. Следовательно, $\max(x, x + 1, \ldots, x + k - 1) = k - 1$. Так мы получили константу $k - 1$. Навешивая на неё отрицания Поста, получим остальные константы. Теперь покажем, как строить все функции от одной переменой. Будем искать отрицание Лукашевича (с помощью которого из $\max$ получим $\min$) и все функции $J_i$.

    Снова рассмотрим максимум выражений $x, x + 1, \ldots, x + k - 1$, но на этот раз отбросим одно из них --- например, $x + j$. Теперь максимум от оставшихся есть либо $k - 1$, либо $k - 2$. Принимая во внимание $x + j = k - 1 \Leftrightarrow x = k - j - 1$, получаем, что функция $\varphi_j(x) \vcentcolon = \max(x + \alpha : \alpha = 0, 1, \ldots, j - 1, j + 1, \ldots, k - 1)$ равна $k - 1$ при $x \ne k - 1 - j$ и $k - 2$ при $x = k - 1 - j$. Прибавив к этой функции $1$, получим $J_{k - j - 1}(x)$. Т.\,к. $j$ можно варьировать на множестве $\{0, 1, \ldots, k - 1\}$. то в результате имеем все функции $J_0(x), \ldots, J_{k - 1}(x)$. Мы будем получать произвольные функции одной переменой, <<складывая>> их из отдельных <<столбиков>>. У нас уже имеются столбики высоты $k - 1$ --- графики функций $J_i$. Если взять $\max(J_i, p)$ ($p = 0, 1, \ldots, k - 1$), то получится функция, принимающая в точке $i$ значение $k - 1$, а в остальных --- значение $p$. Прибавив $k - p$, получаем один <<столбик>> высоты $k - p - 1$. Обозначим $h_{p, i} \vcentcolon = \max(J_i, p) + (k - p)$. Варьируя $p$, получаем все функции $f_{s, i}(x) \vcentcolon = h_{k - 1 - s, i}(x)$. Теперь $\forall g \in P_k^1$ имеем
    \[
        g(x) = \max(f_{g(0), 0}(x), \ldots, f_{g(k - 1), k - 1}).
    \]
    Таким образом получаем отрицание Лукашевича, а из него и $\max$ --- $\min$. Теперь имеем всю полную систему $\{0, 1, \ldots, k - 1, J_0(x), \ldots, J_{k - 1}(x), \min(x_1, x_2), \max(x_1, x_2)\}$.
\end{proof}

\begin{definition}
    $V_k(x_1, x_2) \vcentcolon = \max(x_1, x_2) + 1$ --- \textit{функция Вебба}.
\end{definition}

\begin{theorem}
    Система $\{V_k(x_1, x_2)\}$ полна в $P_k$.
\end{theorem}

\begin{proof}
    Из функции Вебба получается отрицание Поста $V_k(x, x) = x + 1 = \overline{x}$. Следовательно, получаем функции $x + i$ ($i = 0, \ldots, k - 1$). Теперь получаем $\max(x_1, x_2) = V_k(x_1, x_2) + (k - 1)$. Имеем всю полную систему $\{\overline{x}, \max(x_1, x_2)\}$.
\end{proof}

