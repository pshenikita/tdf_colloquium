\section{Теорема Кузнецова о полноте в $P_k$}

\begin{theorem}[Кузнецов]
    Можно построить систему $M_1, \ldots, M_s$ замкнутых классов $k$-значной логики такую, что ни один из них не содержится в других, причём произвольная система $F \subseteq P_k$ полна тогда и только тогда, когда она не содержится ни в одном из классов $M_1, \ldots, M_s$.
\end{theorem}

В доказательстве под <<леммой $j$>> подразумевается лемма из вопроса номер $20 + j$ ($j = 1, 2, 3$).

\begin{proof}
    Рассмотрим все классы $N_1, \ldots, N_q$ вида $U(K)$, где $K$ --- множество функций от двух переменных, содержащее обе селекторные функции и не содержащее функции Вебба. По лемме 1 они замкнуты. Если система $F \subseteq P_k$ неполна, то по лемме 3 существует такой класс $N_i$, что $F \subseteq N_i$. Если для некоторого класса $N_i$ имеет место $F \subseteq N_i$, то по лемме 2 система $F$ неполна. Таким образом, полнота системы $F$ эквивалентна невключению её ни в один из классов $N_1, \ldots, N_q$. Удалив из системы $N_1, \ldots, N_q$ те классы, которые содержатся в других, получим искомую систему $M_1, \ldots, M_s$.
\end{proof}

\begin{remark}
    Заметим, что $M_1, \ldots, M_s$ --- все предполные классы в $P_k$. Это доказывается так же, как и в случае $P_2$.
\end{remark}

