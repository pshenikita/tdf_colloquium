\section{Функции и множества. Равенство функций. $n$-местные функции. Функции алгебры логики, из задание таблицами. Число $n$-местных функций алгебры логики. Существенные и несущественные переменные. Операция добавления либо удаления несущественной переменной. Симметрические функции алгебры логики. Элементарные функции алгебры логики}

\begin{definition}
    Чтобы задать функцию $f$, нужно забать, во-первых, множество $\Dom(f)$  --- \textit{область определения} функции, а во-вторых, для каждого элемента $x$ области определения указать \textit{значение} $f(x)$ этой функции.
\end{definition}

\begin{definition}
    $f = g \overset{\mathrm{def}}{\Longleftrightarrow} \br{\Dom(f) = \Dom(g)} \wedge \br{\forall x \in \Dom(f)\; f(x) = g(x)}$.
\end{definition}

\begin{definition}
    \textit{Упорядоченный набор} $(a_1, \ldots, a_n)$ длины $n \in \N$ --- отображение из $\{1, \ldots, n\}$, принимающее в точке $i$ значение $a_i$. Иногда рассматривают набор длины $0$. Он определён на пустом множестве и обозначается через $\Lambda$.
\end{definition}

\begin{definition}[Прямое произведение]
    $A_1 \times \ldots \times A_n \vcentcolon = \{(a_1, \ldots, a_n) : a_i \in A_i, i = 1, \ldots, n\}$.
\end{definition}

\begin{definition}
    Если $\Dom(f) = A_1 \times \ldots \times A_n$ (для некоторого набора множеств $\{A_i\}_{i = 1}^n$), то функцию $f$ называют \textit{$n$-местной} (\textit{функцией от $n$ переменных}). Обозначается через $f(x_1, \ldots, x_n)$ или $f(\widetilde{x}^n)$.
\end{definition}

\begin{definition}
    $E_2 \vcentcolon = \{0, 1\}$, $B_n \vcentcolon = E_2^n$ --- \textit{$n$-мерный булев куб}.
\end{definition}

\begin{proposal}
    $\abs{B_n} = 2^n$.
\end{proposal}

\begin{proof}
    Любой элемент $B_n$ имеет вид $(a_1, \ldots, a_n)$, где $a_i = 0$ или $a_i = 1$ $\forall i = 1, \ldots, n$. Тогда для каждой из $n$ координат имеет ровно $2$ значения, значит, всего значений $2^n$.
\end{proof}

\begin{remark}
    $B_n$ представляет собой множество вершин $n$-мерного гиперкуба со стороной $1$. Допускается случай $n = 0$, и тогда булев куб состоит из единтвенного пустого набора $\Lambda$.
\end{remark}

\begin{definition}
    \textit{Функцией алгебры логики} называется функция $f: B_n \to E_2$.
\end{definition}

\begin{remark}
    Эта функция задаёт раскраску вершин $n$-мерного гиперкуба в два цвета.
\end{remark}

Множество всех функций алгебры логики обозначается через $P_2$. Множество всех $n$-местных функций алгебры логики обозначается через $P_2^n$.

Функцию алгебры логики $f(\widetilde{x}^n)$ можно задать таблицей, имеющей $n + 1$ столбец. В первых $n$ столбцах, соответствующих переменным $x_1, \ldots, x_n$, перечисляются все $2^n$ возможных наборов их значений. В последнем столбце указываются соответствующие значения функции $f$.

\begin{proposal}
    $\abs{P_2^n} = 2^{2^n}$.
\end{proposal}

\begin{proof}
    Высота каждого столбца таблицы равна $2^n$ ($ = \abs{B_n}$), а в правом столбце мы ставим значения функции на каждом из наборов в первых $n$ столбцах таблицы, т.\,е. элемент $B_{2^n}$, которых ровно $2^{2^n}$.
\end{proof}

\begin{definition}
    Функцию алгебры логики $f(\widetilde{x}^n)$ назовём \textit{существенно зависящей от переменной $x_i$} ($i = 1, \ldots, n$), если существуют значения $\alpha_1, \ldots, \alpha_{i - 1}, \alpha_{i + 1}, \ldots, \alpha_n$ из $E_2$ такие, что \[f(\alpha_1, \ldots, \alpha_{i - 1}, 0, \alpha_{i + 1}, \ldots, \alpha_n) \ne f(\alpha_1, \ldots, \alpha_{i - 1}, 1, \alpha_{i + 1}, \ldots, \alpha_n).\]
    В этом случае $x_i$ называется \textit{существенной переменной функции $f$}. Переменная, не являющаяся существенной. называется \textit{фиктивной}.
\end{definition}

\begin{definition}
    Пусть $x_i$ --- фиктивная переменная функции $f(\widetilde{x}^n)$. Тогда функция \[g(x_1, \ldots, x_{i - 1}, x_{i + 1}, \ldots, x_n) \vcentcolon = f(x_1, \ldots, x_{i - 1}, 0, x_{i + 1}, \ldots, x_n)\]
    называется \textit{полученной из $f$ удалением фиктивной переменной $x_i$}. Обратно, говорят, что \textit{$f$ получена из $g$ добавлением $i$-ой фиктивной переменной}.
\end{definition}

\begin{definition}
    Если функции $f$ и $g$ получены друг из друга цепочкой добавлений и удалений фиктивных переменных, то назовём их \textit{эквивалентными}.
\end{definition}

\begin{definition}
    Функция алгебры логики $f(\widetilde{x}^n)$ называется \textit{симметрической относительно переменных $x_{i_1}, \ldots, x_{i_k}$}, если любая перестановка значений этих переменных не изменяет значения функции. В частности, когда функция является симметрической относительно всех своих переменных, она называется просто \textit{симметрической}.
\end{definition}

Такую функцию можно задавать таблицей, существенно более короткой, чем в общем случае. Она имеет всего два столбца: в первом указывается количество переменных равных $1$, во втором --- соответствующие значения функции. Таблица имеет всего $n + 1$ строку.

\begin{definition}
    \textit{Элементарными} мы будем называть следующие функции:
    \begin{enumerate}[nolistsep]
        \item \textit{Константы} $0$ и $1$ (нуль-местные функции).
        \item \textit{Тождественная функция} $x$ и \textit{отрицание} $\overline{x} \vcentcolon = 1 - x$ (одноместные функции).
        \item \textit{Конъюнкция} $x_1 \& x_2 \vcentcolon = \min\{x_1, x_2\}$ (иногда обозначается как $x_1 \cdot x_2$).
        \item \textit{Дизъюнкция} $x_1 \vee x_2 \vcentcolon = \max\{x_1, x_2\}$.
        \item \textit{Имплкация} $x_1 \to x_2$, $x_1 \to x_2 = 0 \overset{\mathrm{def}}{\Longleftrightarrow} x_1 = 1, x_2 = 0$.
        \item \textit{Сумма по $\bmod$ $2$} $x_1 \oplus x_2$, $x_1 \oplus x_2 = 0 \overset{\mathrm{def}}{\Longleftrightarrow} x_1 = x_2$.
        \item \textit{Эквивалентность} $x_1 \sim x_2$, $x_1 \sim x_2 = 1 \overset{\mathrm{def}}{\Longleftrightarrow} x_1 = x_2$.
        \item \textit{Штрих Шеффера} $x_1 \mid x_2 \vcentcolon = \overline{x_1 \cdot x_2}$.
        \item \textit{Стрелка Пирса} $x_1 \downarrow x_2 \vcentcolon = \overline{x_1 \vee x_2}$.
    \end{enumerate}
\end{definition}

