\section{Теорема о выделении из полной системы функций алгебры логики полной подсистемы, имеющей не более 4 функций}

\begin{theorem}
    Из каждой полной системы функций алгебры логики можно выделить полную подсистему, имеющую не более 4 функций.
\end{theorem}

\begin{proof}
    Пусть система $F$ полна. Рассмотрим функции $\{f_1, \ldots, f_5\}$, введённые при доказательстве теоремы Поста. Они образуют полную подсистему. Если $f_1 \in T_1$, то $f_1$ несамодвойственная и можно было взять $f_5 = f_1$. Если же $f_1 \notin T_1$, то $f_1$ немонотонна ($f_1 \ne 0$, т.\,к. $f_1 \notin T_0$) и можно взять $f_4 = f_1$.
\end{proof}

\begin{remark}
    Данная оценка точна --- примером полной подиситемы из четырёх функций, теряющей свойство полноты при удалении любой из них, служит $F = \{0, 1, x_1x_2, x_1 \oplus x_2 \oplus x_3\}$.
\end{remark}

