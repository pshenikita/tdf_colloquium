\section{Предполные классы. Предполнота классов $T_0$, $T_1$, $L$, $S$, $M$. Отсутствие в $P_2$ других предполных классов}

\begin{definition}
    Класс $K$ функций алгебры логики называется \textit{предполным}, если $[K] \ne P_2$, но $\forall f \in P_2 \setminus K$ выполнено $[K \cup \{f\}] = P_2$.
\end{definition}

\begin{proposal}
    Любой предполный класс замкнут.
\end{proposal}

\begin{proof}
    Предположим противное. Рассмотрим $f \in [K] \setminus K$. $[K \cup \{f\}] = P_2$. С другой стороны, $K \cup \{f\} \subseteq [K]$, а значит, $[K \cup \{f\}] \subseteq [[K]] = [K] \ne P_2$. Противоречие.
\end{proof}

\begin{corollary}
    В алгебре логики имеются ровно 5 предполных классов: $T_0$, $T_1$, $L$, $S$, $M$.
\end{corollary}

\begin{proof}
    Действительно, если класс $K$ предполон, то он, согласно следствию 1 из предыдущего вопроса, должен содержаться в одном из классов $T_0$, $T_1$, $L$, $S$, $M$. Обозначим его за $Q$. Если бы он не совпал с классом $Q$, то можно было бы взять функцию $f \in Q \setminus K$. Т.\,к. $[K \cup \{f\}] \subseteq [Q] = Q \ne P_2$, то получили бы противоречие с предполнотой класса $K$.

    Обратно, рассмотрим произвольный класс $Q$ из указанных пяти классов. Возьмём произвольную функцию алгебры логики $f$, не принадлежащую $Q$. Рассмотрим класс $Q^\prime = [Q \cup \{f\}]$. Если бы он не совпал с $P_2$, то должен был бы содержаться в одном из классов $T_0$, $T_1$, $S$, $M$, $L$, причём (из-за добавленной функции $f$) отличном от класса $Q$. Однако ни один из данных классов не содержится в другом. Поэтому класс $Q^\prime$ совпадает с $P_2$, т.\,е. класс $Q$ предполон.
\end{proof}

