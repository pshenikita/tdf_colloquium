\section{Замыкание множества функций алгебры логики. Примеры. Простейшие свойства замыкания. Замкнутые классы. Примеры}

\begin{definition}
    \textit{Замыканием} $[M]$ множества $M$ функций алгебры логики называется множество всех функций, которые можно получить суперпозициями над $M$.
\end{definition}

\begin{definition}
    Множество $M$ называется \textit{замкнутым}, если $[M] = M$.
\end{definition}

\begin{proposal}[Простешие свойства замыкания]
    \begin{enumerate}[nolistsep]
        \item $[M] \supseteq M$;
        \item $[[M]] = [M]$ (<<замыкание замкнуто>>);
        \item $M_1 \subseteq M_2 \Rightarrow [M_1] \subseteq [M_2]$;
        \item $[M_1 \cup M_2] \supseteq [M_1] \cup [M_2]$.
    \end{enumerate}
\end{proposal}

Переформулировка полноты множества функций в терминах замыкания: система $M \subseteq P_2$ полна тогда и только тогда, когда $[M] = P_2$. Примеры замкнутых классов функций --- заголовки следующих 4 вопросов. Любая неполная система является подмножеством некоторого отличного от $P_2$ замкнутого класса.

