\section{Простейшие тождества для функций в $P_k$. Аналог совершенной дизъюнктивной нормальной формы для $P_k$}

\begin{enumerate}
    \item Операции $\min(x_1, x_2)$, $\max(x_1, x_2)$, $x_1 \cdot x_2 \pmod k$, $x_1 + x_2 \pmod k$ ассоциативны и коммутативны.
    \item Дистрибутивности: $(x_1 \vee x_2) \& x_3 = (x_1 \& x_3) \vee (x_2 \& x_3)$, $(x_1 \& x_2) \vee x_3 = (x_1 \vee x_3) \& (x_1 \vee x_3)$, $(x_1 + x_2) \cdot x_3 = x_1 \cdot x_3 + x_2 \cdot x_3$.
    \item При $k > 2$: ${\sim}({\sim}x) = x$, $\overline{\overline{x}} = x + 2$.
    \item ${\sim}(\min(x_1, x_2)) = \max({\sim}x_1, {\sim}x_2)$ --- аналог закона де Моргана. Заметим, что для отрицания Поста аналогичное равенство при $k > 3$ тождеством не является.
\end{enumerate}

Используя операции $J_i$, $\vee$, $\&$ и константы, можно построить аналог СДНФ в $k$-значной логике. Именно, для произвольной функции $f \in P_k^n$ имеет место следующее тождество:
\[
    f(x_1, \ldots, x_n) = \bigvee_{(\sigma_1, \ldots, \sigma_n) \in E_k^n}J_{\sigma_1}(x_1) \& \ldots \& J_{\sigma_n}(x_n) \& f(\sigma_1, \ldots, \sigma_n).
\]
Доказывается так же, как и в случае СДНФ.

