\section{Алгоритм распознавания полноты в $P_k$. Последовательность Кузнецова для множества $F$ функций $k$-значной логики. Стабилизация этой последовательности на множестве $[F]_{x_1x_2}$}

\begin{definition}
    Индукцией по определению формулы в сигнатуре $\Sigma: S \to F$ ($F \subseteq P_k$) определим \textit{глубину формулы}:
    \begin{enumerate}
        \item Если $x_i$ --- символ переменной, то глубина формулы $x_i$ равна $0$.
        \item Если $s \in S$, функция $f = \Sigma(s)$ зависит от $n$ переменных, $\Phi_1, \ldots, \Phi_n$ --- формулы в сигнатуре $\Sigma$, причём $m$ --- наибольшая из глубин этих формул, то глубина формулы $s(\Phi_1, \ldots, \Phi_n)$ равна $m + 1$.
    \end{enumerate}
    Глубину формулы $\Phi$ будем обозначать через $d(\Phi)$.
\end{definition}

\begin{remark}
    Очевидно, что глубина формулы --- это максимальная из длин цепочек <<вложенных>> друг в друга <<операций>> для неё.
\end{remark}

\begin{theorem}
    Существует алгоритм, распознающий полноту конечных систем функций в $P_k$.
\end{theorem}

\begin{definition}
    Определим последовательность $G_1, G_2, \ldots$ множеств функций в $P_k$, зависящих от двух переменных. В качестве $G_i$ берётся множество всех функций, определяемых в сигнатуре $\Sigma$ невырожденными формулами, содержащими только переменные $x_1$, $x_2$ и имеющими глубину, меньшую $i$. При определении функции каждая такая формула будет рассматриваться относительно переменных $x_1$, $x_2$. Эту последовательность назовём \textit{последовательностью Кузнецова}.
\end{definition}

Очевидно, $\varnothing = G_1 \subseteq G_2 \subseteq \ldots$ Т.\,к. число всех функций в $P_k$, зависящих от двух переменных, равно $k^{k^2}$, то $\abs{G_i} \leqslant k^{k^2}$ для всех $i$, и указанная последовательность стабилизируется, начиная с некоторого номера $m$: $G_m = G_{m + 1} = \ldots = G$.

\begin{definition}
    Будем называть $G$ \textit{пределом последовательности Кузнецова}.
\end{definition}

При построении последовательности Кузнецова будем связывать с каждой функцией $G_i$ некоторую формулу, содержащую только переменные $x_1$, $x_2$ и имеющую глубину, меньшую $i$, реализующую эту функцию. Рассмотрим произвольную функцию $f$ из $G_{i + 1}$, не вошедшую в $G_i$. Она должна определяться некоторой формулой $\Phi$ вида $s(\Phi_1, \ldots, \Phi_n)$, содержащей только переменные $x_1$, $x_2$ и имеющей глубину меньше $i + 1$. Тогда глубина каждой формулы $\Phi_j$ меньше $i$, т.\,е. она является либо переменной, либо определяет функцию $g_j \in G_i$. Но с этими функциями при построении $G_i$ мы уже связали какие-то определяющие их формулы $\Phi_j^\prime$, возможно, отличающиеся от формул $\Phi_j$. Очевидно, что при замене в формуле $\Phi$ отличных от переменных формул $\Phi_j$ на реализующие те же самые функции формулы $\Phi_j^\prime$ мы получим формулу $\Phi^\prime$, определяющую ту же самую функцию, что и формула $\Phi$.

Это означает, что для получения $G_{i + 1}$ из $G_i$ нам достаточно рассмотреть только всевозможные формулы $\Phi^\prime$ вида $s(\Phi_1^\prime, \ldots, \Phi_n^\prime)$, где $s \in S$, а $\Phi_1^\prime, \ldots, \Phi_n^\prime$ --- переменные либо формулы, сопоставленные функциям из $G_i$. Те из формул $\Phi^\prime$, которые дадут новые функции, не встречавшиеся в $G_i$, сопоставляются этим функциям, а сами функции добавляются к $G_i$. Повторения отбрасываются.

Таким образом, получаем алгоритм построения множеств $G_i$ ($i = 1, 2, \ldots$). Т.\,к. $G_{i + 1}$ однозначно определяется по $G_i$, эту последовательность достаточно прослеживать до первого совпадения её членов, далее она стабилизируется. Очевидно, первое совпадение произойдёт за конечное число шагов, т.\,е. получаем алгоритм нахождения предела $G$ последовательности Кузнецова.

\begin{proof}
    Предположим, что функция Вебба $V_k(x_1, x_2)$ принадлежит к пределу $G$. Тогда, по определению последовательности Кузнецова, она задаётся формулой в сигнатуре $\Sigma$, т.\,е. получается суперпозициями из $F$, и $F$ полно.

    Обратно, пусть $F$ полно. Тогда функция Вебба задаётся некоторой формулой в сигнатуре $\Sigma$, имеющей ровно две существенные переменные. Переобозначим эти переменные на $x_1$, $x_2$, а все несущественые переменные заменим на $x_1$. Если эту формулу рассматривать относительно $x_1$, $x_2$, то она снова будет определять функцию Вебба. Однако, по нашему определению, такая формула определяет функцию из множества $G_{i + 1}$, где $i$ --- глубина формулы. Следовательно, функция Вебба попадает в множество $G$.

    Итак, алгоритм распознавания полноты конечной системы функций в $P_k$ состоит в построении по ней последовательности Кузнецова и проверке вхождения функции Вебба в предел этой последовательности. Если входит, то система полна, иначе --- неполна.
\end{proof}

