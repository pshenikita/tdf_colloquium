\section{Формулы алгебры логики. Слова в конечных алфавитах. Сигнатура. Определение формулы в сигнатуре $\Sigma$. Значение формулы $\Phi$ на наборе $\widetilde{\alpha}$ значений переменных $\widetilde{x}$. Существенные и несущественные переменные формулы. Функция, определяемая формулой $\widetilde{\Phi}$ относительно переменных $\widetilde{x}$. Функция, получаемая суперпозициями над множеством функций $F$. Определение суперпозиций, не использующее понятия формулы (без доказательства эквивалентности). Операции суперпозиции: подстановка переменных, подстановка функции, добавление либо удаление несущественных переменных}

\begin{definition}
    \textit{Алфавитом} может служить любое множество, а \textit{словом} в этом алфавите называется упорядоченный набор $\alpha$ элементов из $A$. Обычно будем пользоваться записью $\alpha(1)\ldots\alpha(n)$. Пустое слово обозначается через $\Lambda$.
\end{definition}

Пусть $S$ --- множество символов (на самом деле, объектов произвольной природы), которые будут использоваться для обозначения функций из $F$ --- множества функций алгебры логики, которые мы хотим считать <<элементарными>>.

\begin{definition}
    Отображение $\Sigma : S \to F$ назовём \textit{сигнатурой} для $F$.
\end{definition}

\begin{remark}
    Различным символам из $S$ может сопоставляться одна и та же функция.
\end{remark}

Для построения формул нам понадобятся также переменные. Выберем фиксированное счётное множество $X \vcentcolon = \{x_1, x_2, \ldots\}$, которые будем называть \textit{символами переменных}.

\begin{definition}
    \textit{Формулы в сигнатуре $\Sigma$} определяются индуктивно.
    \begin{enumerate}
        \item Если $x_i$ --- символ переменной, то однобуквенное слово, образованное символом $x_i$ --- формула в $\Sigma$.
        \item Если функция $f = \Sigma(s)$ (где $s \in S$) зависит от $n$ переменных, причём $\Phi_1, \ldots, \Phi_n$ --- формулы в сигнатуре $\Sigma$, то слово $s(\Phi_1, \ldots, \Phi_n)$ --- формула в сигнатуре $\Sigma$.
    \end{enumerate}
\end{definition}

Таким образом, каждая формула в сигнатуре $\Sigma$ представляет собой слово в алфавите $X \cup S \cup \{\text{<<,>>}, \text{<<(>>}, \text{<<)>>}\}$. Отметим, что пока мы никак не определили связь между формулами и функциями. Мы лишь дали индуктивное определение некоторому классу слов.

Пусть $\Phi$ --- формула в сигнатуре $\Sigma$; $\widetilde{x} = (x_{i_1}, \ldots, x_{i_n})$ --- какой-либо упорядоченный набор переменных, включающий все переменные формулы $\Phi$; $\widetilde{\alpha} = (\alpha_1, \ldots, \alpha_n)$ --- двоичный набор ($\alpha_i \in E_2$, $i = 1, \ldots, n$).

\begin{definition}
    Определим \textit{значение формулы $\Phi$ на наборе $\widetilde{\alpha}$} переменных $\widetilde{x}$ индукцией по построению формулы $\Phi$:
    \begin{enumerate}[nolistsep]
        \item Если $\Phi$ есть однобуквенное слово $x_{i_j}$, то $\Phi[\widetilde{x}, \widetilde{\alpha}] \vcentcolon = \alpha_j$.
        \item Пусть $\Phi$ имеет вид $s(\Phi_1, \ldots, \Phi_n)$, $f = \Sigma(s)$, причём $\Phi_1[\widetilde{x}, \widetilde{\alpha}] = \beta_1, \ldots, \Phi_m[\widetilde{x}, \widetilde{\alpha}] = \beta_m$. Тогда $\Phi[\widetilde{x}, \widetilde{\alpha}] \vcentcolon = f(\beta_1, \ldots, \beta_m)$.
    \end{enumerate}
\end{definition}

Заметим, что функции по формулам мы всё ещё не определили.

\begin{definition}
    Переменную $x_{i_j}$ формулы $\Phi$ назовём \textit{существенной}, если существуют такие значения $\alpha_1, \ldots, \alpha_{i - 1}, \alpha_{i + 1}, \ldots, \alpha_n$, что
    \[
        \Phi[\widetilde{x}, \alpha_1, \ldots, \alpha_{i - 1}, 0, \alpha_{i + 1}, \ldots, \alpha_n] \ne \Phi[\widetilde{x}, \alpha_1, \ldots, \alpha_{i - 1}, 1, \alpha_{i + 1}, \ldots, \alpha_n].
    \]
    Иначе переменная $x_{i_j}$ \textit{фиктивная}.
\end{definition}

\begin{definition}
    Пусть $\Phi$ --- формула в сигнатуре $\Sigma$; $P = \{x_{i_1}, \ldots, x_{i_n}\}$ --- некоторое множество переменных, включающее все существенные переменные формулы $\Phi$. Будем считать, что $i_1 < \ldots < i_n$. Обозначим $\widetilde{x} \vcentcolon = (x_{i_1}, \ldots, x_{i_n})$. Рассмотрим $f \in P_2^n$ такую, что $\forall \widetilde{\alpha} = (\alpha_1, \ldots, \alpha_n) \in B_n$ выполнено $f(\alpha_1, \ldots, \alpha_n) = \Phi[\widetilde{x}, \widetilde{\alpha}]$. Тогда \textit{формула $\Phi$ определяет функцию $f$ относительно переменных $P$}.
\end{definition}

\begin{definition}
    Будем называть формулы в сигнатуре $\Sigma$, представляющие собой переменные, \textit{вырожденными}, а прочие формулы --- \textit{невырожденными}. Если функция $f$ определена в сигнатуре $\Sigma: S \to F$ невырожденной формулой, то говорим, что она \textit{получена суперпозициями над $F$}.
\end{definition}

Суперпозиции можно определить и без понятия формулы.

\begin{definition}
    Функция тогда и только тогда \textit{получена суперпозициями над $F$}, когда она может быть получена из функций системы $F$ конечной последовательностью применений следующий операций:
    \begin{enumerate}[nolistsep]
        \item \textbf{Операция подстановки переменных}. Пусть $f(\widetilde{x}^n) \in P_2$. Рассмотрим упорядоченный набор $(i_1, \ldots, i_n)$, элементов множества $\{1, \ldots, n\}$ (не обязательно перестановку). Пусть $g(\widetilde{x}^n)$ --- функция, определённая на $B_n$, причём $g(x_1, \ldots, x_n) \vcentcolon = f(x_{i_1}, \ldots, x_{i_n})$. Тогда скажем, что $g$ получена из $f$ \textit{операцией подстановки переменных}.
        \item \textbf{Операций подстановки одной функции в другую}. Пусть имеются функции $f \in P_2^n$ и $g \in P_2^m$. Рассмотрим функцию $h \in P_2^{n + m - 1}$ такую, что
            \[
                h(x_1, \ldots, x_{m + n - 1}) \vcentcolon = f(x_1, \ldots, x_{n - 1}, g(x_n, \ldots, x_{n + m - 1})).
            \]
            Скажем, что $h$ получена из функций $f$ и $g$ \textit{операций подстановки одной функции в другую}.
        \item \textbf{Операция добавления или удаления фиктивной переменной}.
    \end{enumerate}
\end{definition}

Эквивалентность определений доказывается в курсе математической логики.

