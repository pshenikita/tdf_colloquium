\section{Представление функций в $P_k$ полиномами}

\begin{theorem}
    Система полиномов по модулю $k$ полна в $P_k$ тогда и только тогда, когда $k$ --- простое число.
\end{theorem}

\begin{proof}
    Рассмотрим произвольную функцию $f(x_1, \ldots, x_n)$ из $P_k$. Для неё имеет место следующее сотношение:
    \[
        f(x_1, \ldots, x_n) = \sum_{(\sigma_1, \ldots, \sigma_m)}j_{\sigma_1} \cdot \ldots \cdot j_{\sigma_n} \cdot f(\sigma_1, \ldots, \sigma_n) \pmod k.
    \]
    Это аналог СДНФ из алгебры логики. Вместо максимума используется сумма по модулю $k$, а вместо минимума --- умножение по модулю $k$.

    Тождество $j_\sigma(x) = j_0(x - \sigma)$ позволяет выразить все функции $j_\sigma$ через функцию $j_0$. Если функцию $j_0$ можно выразить полиномом по модулю $k$, то, подставляя это выражение в указанную выше формулу, раскрывая скобки и приводя подобные члены, получим представление полиномом произвольной функции $f$. Обратно, если $j_0$ не выразима полиномом, то не каждая функция из $P_k$ выразима (т.\,к. $j_0 \in P_k$). Здесь рассмотрим два случая:
    \begin{enumerate}
        \item $k$ простое. Вспомним малую теорему Ферма, согласно которой $x^{k - 1} \equiv 1 \pmod k$ при всех $x = 1, \ldots, k - 1$. Из неё вытекает соотношение $j_0(x) = 1 - x^{k - 1} \pmod k$, т.\,е. $j_0$ представима полиномом по модулю $k$, и система полиномов в этом случае полна.
        \item $k$ составное. Пусть $k = k_1 \cdot k_2$, где $k_1 \geqslant k_2 > 1$ --- натуральные. Предположим, что $j_0 = b_0 + b_1x + \ldots + b_sx^s \pmod k$. Подставляя $x = 0$, получаем $b_0 = 1$. Затем подставляя $x = k_1$, получаем:
            \[
                0 = 1 + b_1k_1 + \ldots + b_sk_1^s \pmod k.
            \]
            Это означает, что для некоторого целого $n$ выполнено:
            \[
                1 + b_1k_1 + \ldots + b_sk_1^s = kn.
            \]
            После перегруппировки членов и замены $k$ на $k_1k_2$ имеем:
            \[
                1 = k_1k_2n - b_1k_1 - \ldots - b_sk_1^s.
            \]
            Правая часть делится на $k_1$, а левая не делится --- противоречие.
    \end{enumerate}
\end{proof}

