\section{Теорема Янова}

\begin{theorem}[Янов]
    Для любого $k \geqslant 3$ в $P_k$ существует замкнутый класс, не имеющий базиса.
\end{theorem}

\begin{proof}
    Напомним, что базисом в замкнутом классе $Q$ называется такая система функций, замыкание которой равно $Q$, а замыкание любой её собственной подсиситемы не равно $Q$. Иными словами, базис --- такая полная в $Q$ система, что любая её функция не выражается суперпозициями через другие функции.

    Рассмотрим последовательность функций из $P_k$:
    \[
        f_0 \vcentcolon = 0,\quad \ldots, \quad
        f_i(x_1, \ldots, x_i) \vcentcolon =
        \begin{cases}
            1,&\text{если $x_1 = \ldots = x_i = 2$}\\
            0,&\text{иначе}
        \end{cases},\quad \ldots;
    \]
    $i = 1, 2, \ldots$ Пусть $M$ --- замыкание множества $\{f_0, f_1, \ldots\}$. Покажем, что оно состоит из функций, отличающихся от функций $f_i$ лишь возможным добавлением фиктивных переменных. Для этого рассмотрим операции суперпозиции, применённые к функциям такого вида:
    \begin{enumerate}
        \item \textbf{Операция подстановки переменных}. Пусть $g(x_1, \ldots, x_n) \vcentcolon = f(x_{i_1}, \ldots, x_{i_n})$. Если $f$ получалась из некоторой $f_j$ добавлением фиктивных переменных, то $g$, ввиду возможного отождествления переменных подстановкой $(i_1, \ldots, i_n)$, будет получаться добавлением несущественных переменных из некоторой функции $f_m$, $m \leqslant i$.
        \item \textbf{Подстановка одной функции в другую}. Пусть функция $h(x_1, \ldots, x_{n + m - 1})$ определена как $f(x_1, \ldots, x_{n - 1}, g(x_n, \ldots, x_{n + m - 1}))$, где $f, g$ получаются из некоторых функций $f_i$ добавлением фиктивных переменных. Т.\,к. $g$ не принимает значение $2$, то $h$ тождественно равна $0$, т.\,е. получается добавлением фиктивных переменных к $f_0$.
        \item \textbf{Добавление либо удаление фиктивной переменной}. Очевидно.
    \end{enumerate}

    Таким образом, класс $M$ полностью описан. Предположим, что он имеет базис $B$ (конечный либо бесконечный) и рассмотрим два случая:
    \begin{enumerate}
        \item В базисе $B$ имеются хотя бы две различных функции. Пусть одна из них получена добавлением фиктивных переменных к $f_i$, другая --- к $f_j$. Если $i = j$, то одна из них может быть получена из другой добавлением либо удалением фиктивных переменных. Если $i < j$, то первая получается из второй отождествлением части переменных с последующим добавлением или удалением фиктивных переменных В обоих случаях одна из функций базиса получается из другой суперпозициями, что невозможно.
        \item В базисе $B$ имеется единственная функция. Пусть она получена добавлением несущественных переменных к $f_i$. Но тогда операциями суперпозиции можно получить только функции, получаемые добавлением фиктивных переменных к функциям $f_j$, $j \leqslant i$. Функцию $f_{i + 1}$ получить нельзя, т.\,е. $B$ --- не базис $M$.
    \end{enumerate}

    Полученное противоречие и доказывает отсутствие базиса в $M$.
\end{proof}

