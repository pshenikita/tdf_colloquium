\section{Неполнота системы $F$, содержащейся в $U(K)$, если $V_k \notin K$}

\begin{lemma}
    Пусть $K$ --- некоторое множество функций от двух переменных, содержащее все селекторные функции от двух переменных и не содержащее функции Вебба. Если $F \subseteq U(K)$, то $F$ неполна.
\end{lemma}

\begin{proof}
    Введём для $F$ какую-либо сигнатуру $\Sigma$ и рассмотрим последовательность Кузнецова $G_1, G_2, \ldots$ Покажем с помощью математической индукции, что для всех $i$ выполнено $G_i \subseteq K$. База: $G_1 = \varnothing \subseteq K$. Пусть уже доказано $G_i \subseteq K$.

    Рассмотрим функцию $h$ из $G_{i + 1}$, не входящую в $G_i$. Она задаётся формулой глубины $i$ в сигнатуре $\Sigma$, имеющей вид $s(A_1, \ldots, A_n)$ ($f \vcentcolon = \Sigma(s) \in F$), где каждое $A_j$ --- либо формула, глубина которой меньше $i$, либо переменная $x_1$, либо переменная $x_2$. В первом случае $A_j$ реализует некоторую функцию $h_j(x_1, x_2)$ из $G_i$. Во втором случае положим $h_j(x_1, x_2) \vcentcolon = g^2_1(x_1, x_2)$, в третьем --- $h_j(x_1, x_2) \vcentcolon = g^2_2(x_1, x_2)$. Т.\,к. $G_i \subseteq K$, то при всех $j = 1, \ldots, n$ получаем $h_j \in K$. Но формула $s(A_1, \ldots, A_n)$ реализует функцию $h(x_1, x_2) = f(h_1(x_2, x_2), \ldots, h_n(x_1, x_2))$. Т.\,к. $f$ сохраняет множество $K$, то и функция $h(x_1, x_2)$ принадлежит $K$. Таким образом, $G_{i + 1} \subseteq K$.

    Из доказанного вытекает, что предел последовательности Кузнецова --- подмножество множества $K$, т.\,е. не содержит функции Вебба. Следовательно, система $F$ неполна.
\end{proof}

