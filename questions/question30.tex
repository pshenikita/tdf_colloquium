\section{Теорема Мучника}

\begin{theorem}[Мучник]
    Для любого $k \geqslant 3$ в $P_k$ существует замкнутый класс, имеющий счётный базис.
\end{theorem}

\begin{proof}
    Рассмотрим последовательность функций $\{f_i(x_1, \ldots, x_i)\}_{i = 1}^\infty$. По определению, функция $f_i(x_1, \ldots, x_i)$ принимает значение $1$, если в наборе значений её аргументов имеется ровно одна единица, а остальные значения равны $2$. В прочих случаях она принимает значение $0$. Обозначим через $M$ замыкание множества $\{f_2, f_3, \ldots\}$. Покажем, что система $\{f_2, f_3, \ldots\}$ является базисом в $M$. Очевидно, она полна в $M$. Покажем, что никакая функция $f_m$, $m \geqslant 2$, не выражается суперпозициями через функции $f_2, \ldots, f_{m - 1}, f_{m + 1}, \ldots$

    Предположим противное. Рассмотрим какую-либо сигнатуру $\Sigma$ для $\{f_2, \ldots, f_{m - 1}, f_{m + 1}, \ldots\}$. Должна существовать невырожденная формула $\Phi$ в данной сигнатуре, определяющая относительно переменных $x_1, \ldots, x_m$ функцию $f_m(x_1, \ldots, x_m)$. Будем считать, что $\Phi$ не имеет других (фиктивных) переменных. Если таковые имелись, вместо них подставляем переменную $x_1$, что не изменит функции, реализуемой формулой. По определению формулы, $\Phi$ имеет вид $s(B_1, \ldots, B_r)$, где $s$ --- символ сигнатуры, обозначающий некоторую функцию $f_r$, $r \ne m$. Каждого $B_i$ --- либо переменная, либо невырожденная формула в $\Sigma$. Рассмотрим произвольный набор $(\alpha_1, \ldots, \alpha_m)$ значений переменных $x_1, \ldots, x_m$. Обозначим через $\beta_i$ значение формулы $B_i$ на данном наборе; $i = 1, \ldots, r$. Тогда $f_m(\alpha_1, \ldots, \alpha_m) = f_r(\beta_1, \ldots, \beta_r)$. Рассмотрим три случая:
    \begin{enumerate}
        \item Среди формул $B_i$ имеется не менее двух невырожденных. Тогда не менее двух значений $\beta_i$ равны $0$ либо $1$. На таком наборе $(\beta_1, \ldots, \beta_r)$ функция $f_r$ обращается в $0$, т.\,е. $f_m$ оказывается тождественно равной нулю, что неверно.
        \item Среди формул $B_i$ имеется ровно одна невырожденная формула; пусть это формула $B_j$. Т.\,к. $r \geqslant 2$, то существует $j^\prime$, отличное от $j$ такое, что $B_{j^\prime}$ --- переменная. Пусть это будет переменная $x_q$. Если положить $\alpha_q = 1$, $\alpha_1 = \ldots = \alpha_{q - 1} = \alpha_{q + 1} = \ldots = \alpha_m = 2$, то $f_m(\alpha_1, \ldots, \alpha_m) = 1$, но $\beta_{j^\prime} = 1$, $\beta_j \in \{0, 1\}$, так что $f_r(\beta_1, \ldots, \beta_r) = 0$ --- противоречие.
        \item Все формулы $B_i$ суть переменные. Тогда $r > m$ (т.\,к. все переменные функции $f_m$ существенные). Следовательно, существуют такие $i$ и $j$ ($i \ne j$) такие, что $B_i$ и $B_j$ --- одна и та же переменная. Пусть это будет переменная $x_q$. Снова берём $\alpha_q = 1$, $\alpha_1 = \ldots = \alpha_{q - 1} = \alpha_{q + 1} = \ldots = \alpha_m = 2$. Тогда $f_m(\alpha_1, \ldots, \alpha_m) = 1$, но $\beta_i = \beta_j = 1$, и $f_r(\beta_1, \ldots, \beta_r) = 0$ --- противоречие.
    \end{enumerate}

    В каждом из случаев получено противоречие, что доказывает теорему.
\end{proof}

\begin{corollary}
    Для любого $k \geqslant 3$ в $P_k$ имеется континуум замкнутых классов.
\end{corollary}

\begin{proof}
    Рассмотрим биекцию $\varphi$ множества рациональных точек отрезка $[0; 1]$ на множество функций $f_2, f_3, \ldots$ из теоремы Мучника. Для любого вещественного числа $a$ из отрезка $[0; 1]$ рассмотрим множество $Q_a$ всех рациональных чисел данного отрезка, меньших или равных числа $a$. Множества $Q_a$ непусты и различны. Их континуум. Отображение $\varphi$ переводит данные множества в различные множества $F_a$ функций последовательности $f_2, f_3, \ldots$ Пусть $M_a$ --- замыкание множества $F_a$. Если $a < b$, то в $F_b$ есть функция, не принадлежащая $F_a$. Согласно доказательству предыдущей теоремы, она не будет принадлежать и замыканию $F_a$. т.\,е. $M_a$. Таким образом, $M_a \ne M_b$, и все замкнутые классы $M_a$ различны. Мощность их множества есть континуум.
\end{proof}

