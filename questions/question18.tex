\section{Замыкание и замкнутые классы в $P_k$. Примеры}

В $k$-значной логике вводятся такие же, как и в алгебре логики, определения замыкания и замкнутого класса.

\begin{example}[Замкнутые классы в $P_k$]
    \begin{enumerate}
        \item $P_k$ --- очевидно.
        \item Этот пример обобщает классы $T_0$ и $T_1$. Пусть $Q \subseteq E_k$. Обозначим через $T_Q$ множество всех функций $f(x_1, \ldots, x_n)$ из $P_k$ таких, что $\forall \alpha_1, \ldots, \alpha_n \in Q$ выполнено $f(\alpha_1, \ldots, \alpha_n) \in Q$. Иными словами, $T_Q$ --- класс функций, сохраняющих множество $Q$. Точно так же, как и для $T_0$ и $T_1$ в $P_2$, доказывается, что $T_Q$ замкнут.
    \end{enumerate}
\end{example}

Рассмотрим $F = \{{\sim}x, \max(x_1, x_2)\}$. Пусть $k \geqslant 3$. Тогда $[F] \subseteq T_{\{0, k - 1\}} \ne P_k$, следовательно, $F$ не полна. Таким образом, замена отрицания Поста на отрицание Лукашевича приводит к потере полноты.

