\section{Полные системы функций алгебры логики. Примеры. Теорема Жегалкина.}

\begin{definition}
    Назовём множество $M$ функций алгебры логики \textit{полным}, если любая функция алгебры логики может быть получена из него суперпозициями.
\end{definition}

\begin{remark}
    Очевидно, что множество $P_2$ полно. Кроме того, согласно доказанной выше теореме, множество $\{\overline{x}_1, x_1x_2, x_1 \vee x_2\}$ тоже полно.
\end{remark}

\begin{proposal}
    Если множество $M_1$ полно, а каждая функция этого множества выражается суперпозициями через функции множества $M_2$, то множество $M_2$ тоже полно.
\end{proposal}

С помощью данного предложения можем привести ещё ряд примеров полных систем:
\begin{enumerate}
    \item $\{\overline{x}_1, x_1x_2\}$ полно, т.\,к. $\{\overline{x_1}, x_1x_2, x_1 \vee x_2\}$ полно и $x_1 \vee x_2 = \overline{\overline{x}_1\overline{x}_2}$;
    \item $\{\overline{x}_1, x_1 \vee y_1\}$ полно --- аналогично;
    \item $\{x_1 \mid x_2\}$ полно, т.\,к. $x_1 \mid x_1 = \overline{x}_1$, а $(x_1 \mid x_2) \mid (x_1 \mid x_2) = \overline{x_1 \mid x_2} = x_1x_2$.
    \item $\{x_1 \downarrow x_2\}$ полно --- аналогично;
    \item $\{0, 1, x_1x_2, x_1 \oplus x_2\}$ полно, т.\,к. $\overline{x} = x \oplus 1$.
\end{enumerate}

\begin{theorem}[Жегалкин]
    Каждая функция алгебры логики представим многочленом по $\bmod$ $2$, причём единственным образом.
\end{theorem}

\begin{proof}
    Существование есть следствие полноты системы $\{0, 1, x_1x_2, x_1 \oplus x_2\}$. Единственность вытекает из того, что полиномов по $\bmod$ $2$ от $n$ переменных столько же, сколько $n$-местных булевых функций. Действительно, каждый полином имеет вид
    \[
        \sum_{\{i_1, \ldots, i_s\} \subseteq \{1, \ldots, n\}}a_{i_1\ldots i_s}x_{i_1}\cdot\ldots\cdot x_{i_s}.
    \]

    Количество членов в указанной сумме есть $2^n$. Для каждого члена имеется две возможности выбора его коэффициентов, отсюда количество полиномов $2^{2^n} = \abs{P_2^n}$.
\end{proof}

