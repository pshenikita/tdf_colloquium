\section{Квадрат в $(E_k)^n$. Лемма о квадрате}

\begin{definition}
    Будем называть \textit{квадратом} любую четвёрку наборов вида
    \[
        \left\{(\alpha_1, \ldots, \alpha_{i - 1}, x, \alpha_{i + 1}, \ldots, \alpha_{j - 1}, y, \alpha_{j + 1}, \ldots, \alpha_n) : x \in \{p_1, p_2\}, y \in \{q_1, q_2\}\right\}.
    \]
\end{definition}

\begin{lemma}[О квадрате]
    Пусть $f(x_1, \ldots, x_n)$ --- существенная функция в $P_k$, принимающая $\ell$ значений, причём $\ell \geqslant 3$. Тогда существует квадрат, на котором $f$ принимает некоторое своё значение ровно в одной точке.
\end{lemma}

\begin{proof}
    Снова без ограничения общности предполагаем, что $x_1$ --- существенная переменная функции $f$. По лемме о трёх наборах, существуют наборы $(\alpha, \alpha_2, \ldots, \alpha_n)$, $(\beta, \alpha_2, \ldots, \alpha_n)$, $(\alpha, \gamma_2, \ldots, \gamma_n)$, на которых $f$ принимает три различных значения. Рассмотрим последовательность:
    \begin{align*}
        P_1 &\vcentcolon = \{(\alpha, \alpha_2, \ldots, \alpha_n), (\beta, \alpha_2, \ldots, \alpha_n)\},\\
        P_2 &\vcentcolon = \{(\alpha, \gamma_2, \alpha_3, \ldots, \alpha_n), (\beta, \gamma_2, \alpha_3, \ldots, \alpha_n)\},\\
        &\vdots\\
        P_i &\vcentcolon = \{(\alpha, \gamma_2, \gamma_3, \ldots, \gamma_i, \alpha_{i + 1}, \ldots, \alpha_n), (\beta, \gamma_2, \gamma_3, \ldots, \gamma_i, \alpha_{i + 1}, \ldots, \alpha_n)\},\\
        &\vdots\\
        P_n &\vcentcolon = \{(\alpha, \gamma_2, \ldots, \gamma_n), (\beta, \gamma_2, \ldots, \gamma_n)\}.
    \end{align*}

    Принцип построения пар следующий: в паре $P_i$ значения $\alpha_2, \ldots, \alpha_i$ заменяются на $\gamma_2, \ldots, \gamma_i$. На наборах пары $P_1$ функция $f$ принимает два различных значения --- $a$ и $b$. На первом наборе пары $P_n$ она принимает значение, отличное от $a$ и $b$. Значение, принимаемое ею на втором элементе пары $P_n$, может совпасть либо с $a$, либо с $b$, но не одновременно с обоими. Следовательно, на паре $P_n$ функция $f$ не принимает хотя бы одно из значений $a$, $b$. Двигаясь по приведённому выше списку пар сверху вниз, мы должны достигнуть такого $i$, что на паре $P_i$ функция $f$ принимает оба значения $a$ и $b$, а на паре $P_{i + 1}$ уже не принимает хотя бы одного из них. Заметим теперь, что объединение пар $P_i$ и $P_{i + 1}$ представляет собой квадрат в $P_k$. То из значений, которое не принимается функцией $f$ на паре $P_{i + 1}$, и будет приниматься ею на указанном квадрате ровно в одной точке.
\end{proof}

