\section{Функции $k$-значной логики. Задание их таблицами, элементарные функции. Формулы $k$-значной логики. Суперпозиция.}

\begin{definition}
    $E_k \vcentcolon = \{0, 1, \ldots, k - 1\}$.
\end{definition}

\begin{definition}
    Функцию $f: E_k \times \ldots \times E_k \to E_k$ будем называть \textit{функцией $k$-значной логики}. Множество всех таких функций обозначается как $P_k$.
\end{definition}

Функцию $k$-значной логики можно задавать таблицей, аналогичной таблице для функции алгебры логики. Т.\,к. каждый аргумент функции имеет ровно одно из $k$ значений, число строк равно $k^n$. Столбец значений имеет ровно столько же позиций, на каждой позиции может быт расположено одно из $k$ значений, т.\,е. число возможных столбцов --- $k^{k^n}$. Т.\,е. $\abs{P_k} = k^{k^n}$.

Как и в случае алгебры логики, выделяется некоторый \textbf{список элементарных функций}:
\begin{enumerate}
    \item \textit{Константы} $0, 1, \ldots, k - 1$ и \textit{тождественная функция} $x$.
    \item $\overline{x} \vcentcolon = x + 1 \pmod k$ --- \textit{отрицание Поста}.
    \item ${\sim}x \vcentcolon = k - 1 - x$ --- \textit{отрицание Лукашевича}
    \item 
        $
        J_i(x) \vcentcolon =
        \begin{cases}
            k - 1,&\text{если $x = i$},\\
            0,&\text{иначе}
        \end{cases}
        $ --- \textit{индикаторная функция}, принимающая в $i$ <<большое значение>>.
    \item 
        $
        j_i(x) \vcentcolon =
        \begin{cases}
            1,&\text{если $x = i$},\\
            0,&\text{иначе}
        \end{cases}
        $ --- \textit{индикаторная функция}, принимающая в $i$ <<маленькое значение>>.
    \item $\min(x_1, x_2)$ --- возможное обобщение конъюнкции (часто будем обозначать через $x_1 \& x_2$).
    \item $x_1 \cdot x_2 \pmod k$ --- другое возможное обобщение конъюнкции.
    \item $\max(x_1, x_2)$ --- возможное обобщение конъюнкции (часто будем обозначать через $x_1 \vee x_2$).
    \item $x_1 + x_2 \pmod k$ --- другое возможное обобщение конъюнкции.
\end{enumerate}

В $k$-значном случае вводятся такие же определения сигнатуры $\Sigma : S \to F$, $F \subseteq P_k$, формулы в сигнатуре $\Sigma$ и функции, реализуемой формулой относительно заданного списка переменных, как и в случае алгебры логики. Дословно так же вводятся понятие суперпозиции, три операции суперпозиции и определение эквивалентности формул.

