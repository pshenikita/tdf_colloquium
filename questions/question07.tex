\section{Классы $T_0$ и $T_1$, их замкнутость}

\begin{definition}
    $T_0 \vcentcolon = \{f \in P_2: f(0, \ldots, 0) = 0\}$ (<<сохраняют $0$>>).
\end{definition}

\begin{proposal}
    Класс $T_0$ замкнут.
\end{proposal}

\begin{proof}
    Проверим на всех операциях суперпозиции:
    \begin{enumerate}
        \item \textbf{Подстановка переменных}. Если $f \in T_0 \cap P_2^n$ и $g \in P_2^n$ получена из $f$ подстановокой переменных $g(x_1, \ldots, x_n) = f(x_{i_1}, \ldots, x_{i_n})$, то $g(0, \ldots, 0) = f(0, \ldots, 0) = 0$.
        \item \textbf{Подстановка функции}. Если $f \in T_0 \cap P_2^n$, $g \in T_0 \cap P_2^m$ и
            \[
                h(x_1, \ldots, x_{n + m - 1}) \vcentcolon = f(x_1, \ldots, x_{n - 1}, g(x_n, \ldots, x_{n + m - 1})),
            \]
            то $h(0, \ldots, 0) = f(0, \ldots, 0, \underbrace{g(0, \ldots, 0)}_{0}) = 0$.
        \item \textbf{Добавление и удаление фиктивных переменных}. Очевидно.
    \end{enumerate}
\end{proof}

\begin{definition}
    $T_1 \vcentcolon = \{f \in P_2: f(1, \ldots, 1) = 1\}$ (<<сохраняют $1$>>).
\end{definition}

\begin{proposal}
    Класс $T_1$ замкнут.
\end{proposal}

\begin{proof}
    Аналогично $T_0$.
\end{proof}

\begin{proposal}
    $\abs{T_0 \cap P_2^n} = \abs{T_1 \cap P_2^n} = 2^{2^n - 1}$.
\end{proposal}

\begin{proof}
    Значение для одного набора уже задано, для остальных выбираются так же, как и раньше.
\end{proof}

