\section{Различие классов $T_0$, $T_1$, $L$, $S$, $M$. Теорема о полноте систем функций алгебры логики}

\begin{theorem}
    Классы $T_0$, $T_1$, $L$, $S$, $M$ различны, и ни один из них не содержится ни в одном другом.
\end{theorem}

\begin{proof}
    Рассмотрим таблицу:
    \[
        \begin{array}{r | c c c c c}
            & T_0 & T_1 & S & M & L\\
            \hline
            0 & + & - & - & + & +\\
            1 & - & + & - & + & +\\
            \overline{x} & - & - & + & - & +\\
            x_1 \cdot x_2 & + & + & - & + & -\\
            x_1 \oplus x_2 & + & - & - & - & +\\
            x_1 \oplus x_2 \oplus x_3 & + & + & + & - & +\\
            \operatorname{Maj}(\widetilde{x}^3) & + & + & + & + & -\\
        \end{array}
    \]
    Здесь $\operatorname{Maj}(x_1, x_2, x_3) \vcentcolon = x_1x_2 \vee x_2x_3 \vee x_3x_1$ --- \textit{функция голосования}. В приведённой таблице для каждой упорядоченной пары классов существует функция, содержащаяся в первом, но не содержащаяся во втором.
\end{proof}

\begin{theorem}[Пост]
    Система функций из $P_2$ полна тогда и только тогда, когда она не содержится ни в одном из классов $T_0$, $T_1$, $L$, $S$, $M$.
\end{theorem}

\begin{proof}
    $\Rightarrow$. Пусть $F$ полна и $F \subseteq K$, где $K \in \{T_0, T_1, L, S, M\}$. Тогда $[F] \subseteq [K] = K \ne P_2$.

    $\Leftarrow$. Обратно, пусть $F \not\subseteq T_0$, $F \not\subseteq T_1$, $F \not\subseteq L$, $F \not\subseteq S$, $F \not\subseteq M$. Тогда в $F$ найдутся $f_1 \notin T_0$, $f_2 \notin T_1$, $f_3 \notin L$, $f_4 \notin S$, $f_5 \notin M$. Возможны два случая:
    \begin{enumerate}
        \item $f_1 \in T_1$. Тогда $\varphi(x) \vcentcolon = f_1(x, \ldots, x) = 1$. Так получаем константу $1$. Чтобы получить константу $0$, достаточно теперь воспользоваться функцией $f_2$.
        \item $f_1 \notin T_1$. Тогда $\varphi(x) \vcentcolon = f_1(x, \ldots, x) = \overline{x}$. По лемме о несамодвойственной функции, из $f_4$ и $\overline{x}$ можно получить константу. Имея отрицание, получаем также и другую константу.
    \end{enumerate}

    Теперь по лемме о немонотонной функции, из $f_5$, $0$, $1$ можно получить $\overline{x}$. По лемме о нелинейной функции, из $f_3$, $0$, $1$ и $\overline{x}$ можно получить $x_1 \cdot x_2$. Следовательно, из $F$ можно выразить полную систему $\{\overline{x}, x_1 \cdot x_2\}$, поэтому $F$ также полна.
\end{proof}

\begin{corollary}
    Каждый замкнутый класс функций из $P_2$, отличный от $P_2$, содержится хотя бы в одном из классов $T_0$, $T_1$, $L$, $S$, $M$.
\end{corollary}

\begin{proof}
    Действительно, если бы он не содержался ни в одном из этих классов, то был бы полон, а т.\,к. замкнут, то совпал бы с $P_2$.
\end{proof}

