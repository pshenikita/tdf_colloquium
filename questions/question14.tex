\section{Полнота относительно замкнутого класса. Базис в замкнутом классе. Примеры. Теоремы Поста (без доказательства) о мощности множества замкнутых классов в $P_2$ и о базисах этих классов}

\begin{definition}
    Пусть $K$ --- замкнутый класс функций алгебры логики. Система $F$ функций этого класса называется \textit{полной в $K$}, если $[F] = K$.
\end{definition}

\begin{definition}
    Система $F \subseteq K$ называется \textit{базисом} в $K$, если она полна в $K$, но каждая её собственная подсистема неполна в $K$.
\end{definition}

\begin{example}
    \begin{enumerate}
        \item $\{x_1 \cdot x_2, 0, 1, x_1 \oplus x_2 \oplus x_3\}$ --- базис в $P_2$.
        \item $\{0, 1, x_1 \cdot x_2, x_1 \vee x_2\}$ --- базис в $M$.

            Если монотонная функция $f$ --- не тождественный $0$, то для неё существует непустое множество $N$ <<нижних единиц>> --- наименьших в смысле рассматриваемого нами частичного порядка наборов, на которых эта функция обращается в $1$. Если она не тождественно равна $1$, то среди её нижних единиц нет нулевого набора. Для произвольной нижней единицы $\alpha$, не являющейся нулевым набором, можно рассмотреть номера $i_1, \ldots, i_k$ всех её разрядов, равных $1$, и построить конъюнкцию $K_\alpha(\widetilde{x}^n) \vcentcolon = x_{i_1}\ldots x_{i_k}$, $k \geqslant 1$. Эта конъюнкция обращается в $1$ на наборе $\alpha$ и на всех больших (в смысле рассматриваемого нами частичного порядка) наборах. Очевидно, что $f$ равна дизъюнкции всех таких конъюнкций:
            \[
                f = \bigvee_{\alpha \in N}K_\alpha(\widetilde{x}^n).
            \]
            Таким образом, если монотонная функция не тождественная константа, то её можно выразить через дизъюнкцию и конъюнкцию. Это и устанавливает полноту рассматриваемой системы функций в $M$. Если удалить из этой системы любую константу, то оставшиеся функции будут сохранять противоположную константу. Если удалить конъюнкцию, то останутся только дизъюнкции. Если удалить последнюю функцию, то через остальные функции можно будет выразить только константы и многоместные конъюнкции. Таким образом, данная система --- базис в $M$.
    \end{enumerate}
\end{example}

\begin{theorem}[Пост]
    Каждый замкнутый класс функций алгебры логики имеет конечный базис.
\end{theorem}

\begin{theorem}[Пост]
    Множество замкнутых классов функций алгебры логики счётно.
\end{theorem}

